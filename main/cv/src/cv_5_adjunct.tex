%%%%%%%%%%%%%%%%%%%%%%%%%%%%%%%%%%%%%%%%%
% Medium Length Professional CV
% LaTeX Template
% Version 2.0 (8/5/13)
%
% This template has been downloaded from:
% http://www.LaTeXTemplates.com
%
% Original author:
% Trey Hunner (http://www.treyhunner.com/)
%
% Important note:
% This template requires the resume.cls file to be in the same directory as the
% .tex file. The resume.cls file provides the resume style used for structuring the
% document.
%
%%%%%%%%%%%%%%%%%%%%%%%%%%%%%%%%%%%%%%%%%

%----------------------------------------------------------------------------------------
%	PACKAGES AND OTHER DOCUMENT CONFIGURATIONS
%----------------------------------------------------------------------------------------

\documentclass{resume} % Use the custom resume.cls style

\usepackage[left=0.75in,top=0.6in,right=0.75in,bottom=0.6in]{geometry} % Document margins
%\usepackage{setspace}

\name{\Large Hossein Akhavan-Hejazi} % Your name
\address{1200 Columbia Ave., Winston Chung Global Energy Center\\ Riverside, CA, 92507} % Your address
%\address{123 Pleasant Lane \\ City, State 12345} % Your secondary addess (optional)
\address{(713)~$\cdot$~471~$\cdot$~9513 \\ shejazi@engr.ucr.edu \\ www.engr.ucr.edu/$\sim$sakhavanhejazi} % Your phone number and email
%\address{ \textbf{www.engr.ucr.edu/~ sakhavanhejazi} }
%\renewcommand{\baselinestretch}{1.5} 

\begin{document}

%----------------------------------------------------------------------------------------
%	EDUCATION SECTION
%----------------------------------------------------------------------------------------

\begin{rSection}{Education}

%{\bf University of California, Berkeley} \hfill {\em June 2004} \\ 
%B.S. in Computer Science \& Engineering \\
%Minor in Linguistics \smallskip \\
%Member of Eta Kappa Nu \\
%Member of Upsilon Pi Epsilon \\
%Overall GPA: 5.678

\textbf{University of California Riverside}, Riverside, CA \\
%\begin{itemize}
 Ph.D.,   Electrical Engineering - Power Systems/ Smart Grid                   \hfill       2016
%        \begin{innerlist}
\begin{itemize}
        \item [] Thesis: \emph{Optimal operation of energy storage units in power transmission and distribution networks.}
\end{itemize}
  

\textbf{Amirkabir University of Technologye}, Tehran, Iran \\
%\begin{outerlist}
%
 M.Sc., Electrical Engineering - Power Systems  \hfill 2011 
%\begin{itemize}
%      %  \item [] Thesis: \emph{Application of phasor measurement unit in measurement-based dynamic load modeling.}
%\end{itemize}


\textbf{Iran University of Science and Technology}, Tehran, Iran \\
 B.Sc., Electrical Engineering \hfill 2008
%\begin{itemize}
%       % \item [] Thesis: \emph{Optimal capacitor allocation in distribution systems using fuzzy expert systems.}
%\end{itemize}




\end{rSection}

%----------------------------------------------------------------------------------------
%	Research INTEREST SECTION
%----------------------------------------------------------------------------------------

\begin{rSection}{Knowledge Areas}
\begin{tabular}{ @{} l  l   } 

\vspace{0.16cm}
Battery Energy Storage: & \begin{tabular}{l}  Battery  operation algorithms,  battery testing,  modeling,   system  design \\ and planning.  \end{tabular} \\  
\vspace{0.16cm}

Smart \& Sustainable grid: & \begin{tabular}{l}   Distribution grid modeling, Distributed Energy Resources,  PV systems. \end{tabular} \\ \vspace{0.16cm}

Power system analysis: & \begin{tabular}{l}  Operation, planning, electricity market operations, risk analysis.   \end{tabular} \\
\vspace{0.16cm}


Optimization:   & \begin{tabular}{l} Stochastic programming, convex relaxations, decomposition.   \end{tabular} \\
\vspace{0.16cm}

Energy Data analysis:&  \begin{tabular}{l} Statistical analysis,   time-series analysis, state estimation. \end{tabular} \\
\end{tabular}
\end{rSection}


%----------------------------------------------------------------------------------------
%	WORK EXPERIENCE SECTION
%----------------------------------------------------------------------------------------
\begin{rSection}{ Experience}
%(Below I have elaborated on my experience most related to the job description)


\begin{rSubsection}{Professional Researcher/  Research Faculty}{ 2017-Present}{Winston Chung Global Energy Center, University of California at Riverside}{}

\vspace{0.26cm}

\item[]  Detailed Modeling of Batteries in Optimal Battery Energy Storage System (BESS) Operation: Variable Cell Balance and Capacity:

\begin{itemize}
   \item  []  The current optimization-based algorithms to operate grid-tied BESS typically do not look much under the hood of the BESS, i.e. the device-level characteristics of the batteries. We propose a new BESS
scheduling optimization framework that accounts for features  such as cell-to-cell variations in (a) maximum capacity, (b) charge level balance, and (c) internal resistance. 
The proposed framework is in the form of tractable mixed integer linear programs (MILP). 
Several  techniques has been  used to convert the original problem in the from of MILP.
%\emph{This project involved developing optimization algorithms, programming in MATLAB and Python and implementing algorithms for estimation and operation control of the BESS.}
\end{itemize}

\item[]  Quantification of Forecast Error Impact on  Model Predictive Control of BESS Operation for Demand Charge Reduction.
\begin{itemize}
   \item  []  The load/generation forecast used in BESS MPC algorithms include some deviations from real-time values. 
 Although the forecast and operating states of the system is frequently updated, such errors have impact on operation efficiency of BESS.
In this work we develop optimization-based mathematical models to quantify the impact on two types of BESS MPC algorithms in demand peak shaving application.  
%\emph{The  algorithm is implemented by programming in MATLAB and Python.}
\end{itemize}

\item[]  Probabilistic Load Forecasting Using Non-parametric Conditional Density Estimation.
\begin{itemize}
  \item  []  In this project we develop and implement algorithms for probabilistic forecasting of the demand.
   The approach is based on estimating the non-parametric kernel-based joint distribution of the demand and affecting factors,e.g. temperature, from historical data.
  The conditional probability distributions are then estimated based on given values of predicted affecting factors.
\end{itemize}


%\begin{itemize}
%   \item  [] Advancing The Resilience And Environmental Performance Of California's Electricity System.
%\end{itemize}
%\item [] UCR PI in  the  proposal submitted in response to CEC-GFO-17-302-G3
%\begin{itemize}
%   \item  [] Demonstrate Business Case for Advanced Microgrids in Support of California's Energy and GHG Policies
%\end{itemize}
%
%\item []UCR Co-PI in  the Willdan Inc.  proposal submitted in response to CEC-GFO-17-302-G1

%\begin{itemize}
\item  [] Optimized Cell Removal to Enhance Operation of the  Used Battery Packs with Variable Capacity Cells.
\begin{itemize}
   \item  []  This project is aimed at developing algorithms for automating and optimizing the BESS asset management. 
  During the operation life of the BESS, the battery pack may develop a condition where the capacity of some cell reduce much greater than the rest of the pack. 
  Th
\end{itemize}




   \item  []  PI,  Co-PI, and key personnel in more than 17 title proposals submitted to CEC, DOE, ARPA-E, UCOP, etc., Including the funded projects:
   \begin{itemize}
  \item  [] Smart Batteries: Self-Healing by Self-Reconfiguration at the Cell Level. (UCR SEED GRANT)
  \item [] Demonstration of Smart Combustion Technology using Natural Gas Fuel Quality Sensor. (CEC-GFO-17-501)
  \item [] Advanced Integrated Building Energy Management Technology Demonstration in a Permanent Supportive Housing Facility. (CEC-GFO-16-309)
  \item [] Internet of Things and Ubiquitous Sensing in University Building Energy Management; Design Optimization and Technology Demonstration. (CEC-GFO-16-309)
  
\end{itemize}
   
   
   
   
%  \end{itemize}
\end{rSubsection}


%%%%%%%%%%%%%%%%%%%%%%%%%%%%%%%%

%\begin{rSubsection}{Post-doctoral Fellow}{June 2017 - September 2017}{Smart Grid Research Lab, University of California at Riverside}{}
%\item []  Developing proposal  for UCOP GFO UC-National Lab Collaborative Research and Training Awards  
%\item []  Developing proposal  for CEC GFO 16-310- Improving Performance and Cost Effectiveness of Wind Energy Technologies 
%\end{rSubsection}
%%%%%%%%%%%%%%%%%%%%%%%%%%%%%%%%%%%%%%%%
%
%\begin{rSubsection}{Post-doctoral Fellow}{April 2017 - June 2017}{System Energy Efficiency  Lab, University of California at San Diego}{}
%
%\vspace{0.26cm}
%
%\item [] Mission planning and task allocation for a group of unmanned aerial vehicles in multi-criteria environmental applications.
%\begin{itemize}
%\item [] Developed optimization-based algorithms to coordinate the mission of a group of UAVs in path finding and identifying the location of fire origination.
%
%\end{itemize}
%
%\end{rSubsection}




\begin{rSubsection}{Post-doctoral Fellow}{2016 - 2017}{Smart Grid Research Lab, University of California at Riverside}{}
\vspace{0.26cm}

\item[]  Utility-scale implementation of a battery-assisted distribution feeder peak-shaving system:

%\begin{innerlist}
\begin{itemize}
%\item [] Analyzed the SCADA data to identify the challenges in standard ToU response of a battery-equipped micro-grid.
\item [] Developed  offline and online  stochastic optimization frameworks for peak-shaving on an industrial 12 kV distribution feeder using batteries located at UCR CE-CERT micro-grid.
%\item  Developed an online model predictive stochastic optimization for peak-shaving.
%\item [] 
Utilized the RPU  SCADA data  to predict the feeder load using ARMA models.%(in offline/online designs)
%\item  Assisted in the implementation of the design on a 1MWh/200 kW Li-ion battery system.
\emph{This project involved extensive programming in MATLAB, data analysis, statistical modelling, Monte-Carlo simulation, and stochastic optimization.}% It also involved developing codes for distribution system modelling and optimal power flow (conic-branch-model).}
\end{itemize}
\item[] Lab-scale implementation of a P-HIL testbed for grid-connected battery systems.
%
\end{rSubsection}

\begin{rSubsection}{Graduate Researcher}{ 2012 - 2016}{Smart Grid Research Lab, University of California at Riverside}{}
\vspace{0.26cm}



\item[] Utility-scale independent energy storage bidding in  electricity market for multiple revenue streams.

%\begin{innerlist}
\begin{itemize}
\item []	Formulated  stochastic optimization  for energy storage bidding in day-ahead electricity market. 
Developed models for participation of independent  energy storage in energy and  reserve market.
%\item 	Developed a mixed integer bender’s decomposition method for storage operation in energy and reserve markets. 
%\item []	
Obtained  convex approximations for the non-linear, non-convex storage operation problem. 
%\item []	
Implemented a stochastic  unit-commitment problem to calculate input bus prices/shadow prices. 
Studied the PJM market structure for demand resources participation in two settlement markets of energy/ancillaries. 
\emph{This project was part of my research towards the PhD dissertation. It involved programming, optimal power flow and unit-commitment modeling,  study of bidding and market clearance procedures, statistical modeling and Monte-Carlo simulations, optimization and decomposition techniques.}
\end{itemize}

%\begin{innerlist}
%\item Investigated the applications of big data analytics in power distribution systems.
%\item  Investigated the applications of μPMU measurements in big data research in power distribution systems.
%\item  Directed an undergraduate researcher on application of HDFS/Hive in analysis of consumer consumption data.
%\end{innerlist}


%\end{innerlist}
%\item   Collaborative research on the applications of big data analytics/ statistics in power systems.

\item[] Energy storage operation in distribution systems via  chance-constrained stochastic programming:

\begin{itemize}
\item [] Developed non-parametric CC-OPF  for energy storage operation in distribution systems.
%\item [] 
Developed   convex approximation of CC-OPF for energy storage operation in distribution systems.
%\item  Implemented several types of stochastic modeling for distribution system analysis in presence of renewable sources.
%\item []  
Developed mathematical models for battery storage characteristics,  analysis of   cost, sizing, and coordinated charging of battery storage.
%
%\item 	Constructed a real measured database of feeder information, load profiles, renewable generation, and vehicle charging. 
%\item [] Developed models for analysis, sizing, and coordinated charging of distributed energy storage on distribution networks. 
%
%\item []  
Developed a test data set for electric vehicle fleet applications in smart grid research. 
\emph{This project involved extensive programming in  MATLAB, distribution grid modeling, and advanced stochastic modelling.}

\end{itemize}



%\begin{innerlist}
%\item Developed models to synthesize the SoC and charging traces for a fleet of PHEV taxi vehicles by combining the GPS data of 536 non-electric taxi vehicles (the Mobility dataset) and the characteristics of five dominant PHEV brands.
%\item Developed models to obtain the loads of charging stations for the PHEV fleet based on the locations and durations that vehicles parked frequently.
%\item 	Developed statistical models for the analysis of both PHEV fleet vehicles and the charging stations.
%\end{innerlist}



%\end{innerlist}


\item[] HIL testing of VAR control in distribution girds via optimal operation of four-quadrant battery chargers.

%\begin{innerlist}
%\begin{itemize}
%\item []	Developed the SOCP and SDP convex relaxation of  OPF on a distribution feeder. 
%%\item []	Developed a semi-definite program (SDP) relaxation of OPF on a distribution test feeder. 
%\item []	Modeled optimal reactive power flow control of distribution systems on real-time digital simulator.
%%\item[] Analyzed the operation results of the 
%%\item 	Connected RTDS Runtime with MATLAB and performed real-time control of grid loads on RTDS. 
%%\item 	Implemented and tested the optimal operation algorithms of EV chargers on RTDS.
%\end{itemize}
%
%
\end{rSubsection}

%------------------------------------------------

\begin{rSubsection}{Graduate Intern }{June 2014 - September 2014}{Energy Management Department, NEC Laboratories America}{}

\item [] Risk-constrained  market optimization and bidding of utility-scale battery storage systems in two-settlement market:

\begin{itemize}
\item []	Developed an optimization for  operation  of battery storage based on NYISO two-settlement market. 
%\item []	
Formulated a cost/revenue model and a revenue- risk model for energy storage system bidding/operation. 
%\item [] 
Obtained tractable models for  battery features such as efficiency and wear cost. 
%\item 	Designed methodologies for calculating marginal cost of energy storage system operation in the market. 
\end{itemize}

\end{rSubsection}

%------------------------------------------------

\begin{rSubsection}{Graduate Researcher }{2011 - 2012}{Electrical Engineering Department, Texas Tech University}{}

\item	[] Distribution system optimal expansion planning with distributed energy resources. 
%\begin{itemize}
%\item [] 	Developed upgrade planing for minimum procurement cost of CHP DG Feed-in-Tariff contracts. 
%\item []	Studied California ISO market structure/ sample utility contracts and collected existing field data. % for most relevant results. 
%\item []	Developed unbalanced power flow tool for distribution systems based on F/B method.
%%\end{innerlist}
%\end{itemize}
\begin{itemize}

\item [] \emph{This project was implemented using MATLAB scripts. Distribution system was modeled via three-phase unbalanced power flow.}
\end{itemize}


\item []	Analysis of the impacts of large scale automated demand response on the electric market operations.

\end{rSubsection}



\begin{rSubsection}{Graduate Researcher}{2009 - 2011}{Electrical Engineering Department, Amirkabir University of Technology}{}

\item []	Modeling dynamic characteristics of distribution systems based on the measurements of PMU.
\item []	Security-constrained co-allocation of energy and reserve in electricity market.
%\begin{itemize}
%\item []	\emph{This is my first work in analysis and modeling of energy market and security constrained optimal dispatch modeling. It resulted in a IEEE transaction paper.}
%\end{itemize}



\end{rSubsection}
\end{rSection}


%%%%%%%%%%%%%%%%%%%%%%%%%%%%%%%%%%%%%%%%%%%%%%%%%%%%%%%%%%%%%%%%%%%%%%%%%%%%%%%%%%%%%
%----------------------------------------------------------------------------------------

\begin{rSection}{Technical Skills}

\begin{tabular}{ @{} l  l   } 
\vspace{0.16cm}

Optimization Software: & \begin{tabular}{l} CPLEX, GUROBI, CVX, MOSEC  \end{tabular} \\ \vspace{0.16cm}

Power System Software:   & \begin{tabular}{l}PSCAD, Simulink, Power World, DIgSILENT, SAM   \end{tabular} \\
\vspace{0.16cm}

Real Time Digital Simulator (RTDS):&  \begin{tabular}{l} Hardware-in-Loop (HIL) Testing \end{tabular} \\
\vspace{0.16cm}

Programming: & \begin{tabular}{l}  MATLAB, Python  \end{tabular} \\
\vspace{0.16cm}

Miscellaneous: &\begin{tabular}{l}  Bash,  Git, LaTeX, MS. Office \end{tabular} \\ 

\end{tabular}

\end{rSection}









%----------------------------------------------------------------------------------------
%	EXAMPLE SECTION
%----------------------------------------------------------------------------------------
%\begin{rSection}{Section Name}
%Section content\ldots
%\end{rSection}
%%%%%%%%%%%%%%%%%%%%%%%%%%%%%%%%%%%%%%%%%%%%%%%%%%%%%%%%%%%%%%%%%%%%%%%%%%%%%%%%%%%%%%%%%%%%%%%%%%%
\begin{rSection}{ Selected Refereed Publications}

{\bf Journal Papers}
%begin{bibsection}
\item [J1]  {\bf H.  Akhavan-Hejazi}, Z. Taylor, ,  H. Mohsenian-Rad, ``Optimal Cell Removal to Enhance Operation of Aged Battery Storage Systems",  submitted. 

\item [J2]  Z. Taylor, {\bf H.  Akhavan-Hejazi},  H. Mohsenian-Rad, ``Optimal Operation of Grid-Tied Energy Storage Systems Considering Detailed Device-Level Battery Models", IEEE Tans. on Industrial Informatics, Online early Access. 

\item [J3]	Y. Zhan, M. Ghamkhari, {\bf H. Akhavan-Hejazi}, D. Xu, H. Mohsenian-Rad, ``Cost-Aware Traffic Management under Demand Uncertainty From a Colocation Data Center User's Perspective", \emph{IEEE Trans. on Services Computing},   January 2018, Online early Access. 

 \item [J4]  Z. Taylor, {\bf H.  Akhavan-Hejazi}, E. Cortez, L.  Alvarez,  S. Ula,  M. Barth,  H. Mohsenian-Rad, ``Customer-side SCADA-assisted Large Battery Operation Optimization for Distribution Feeder Peak Load Shaving",\emph{ IEEE Tans. on Smart Grid}, vol. 10, no. 1, pp 992-1004, January 2019
 
\item [J5] {\bf H. Akhavan-Hejazi}, H. Mohsenian-Rad, ``Power Systems Big Data Analytics: An Assessment of Paradigm Shift, Barriers, and Prospects",   \emph{Energy  Reports},  vol. 4, pp 91-100, November 2018
  

 
 \item [J6] {\bf H. Akhavan-Hejazi},  H. Mohsenian-Rad, ``Energy Storage Planning in Active Distribution Grids: A Chance-Constrained Optimization with Non-Parametric Probability Functions,'' \emph{IEEE Trans. on Smart Grid}, vol. 9, no. 3, pp. 1972-1985, May 2018.




\item [J7] {\bf	H. Akhavan-Hejazi}, H. Mohsenian-Rad,``Optimal Operation of Independent Storage Systems in Energy and Reserve Markets with High Wind Penetration," \emph{IEEE Trans. on Smart Grid}, vol. 5, no. 2, pp. 1088-1097, March 2014.

\item [J8] {\bf	H. Akhavan-Hejazi}, A. Araghi, B. Vahidi, S. Hosseinian, M. Abedi,  H. Mohsenian-Rad, ``Independent Distributed Generation Planning to Profit Both Utility and DG Investors," \emph{IEEE Trans. on Power Systems}, vol. 28, no. 2, pp. 1170-1178, July 2013.

\item [J9]{\bf	H. Akhavan-Hejazi}, H. Mohabati, S. Hosseinian, M. Abedi, ``Differential Evolution Algorithm for Security-Constrained Energy and Reserve Optimization Considering Credible Contingencies," \emph{IEEE Trans. on Power Systems}, vol. 26, pp. 1145-1155, August 2011.



 




%\end{bibsection}
\vspace{+.15275in}

{\bf Book Chapters}

\vspace{+.001275in}

\item [B1]  {\bf H. Akhavan-Hejazi}, H. Mohsenian-Rad,  ``Optimal  Operation of Independent Storage Systems in Energy and Reserve Markets with High Wind Penetration," in \emph{Energy Storage for Smart Grids: Planning \& Operation for Renewable and Variable Energy Resources}, Edited by P. Du and N. Lu, Elsevier, 2014.


\vspace{+.152275in}

%\fontdimen3\font
{\bf   Conference Papers}

\item [C1] Z. Taylor, {\bf H. Akhavan-Hejazi} ,  H. Mohsenian-Rad, ``Power Hardware-in-Loop Simulation of Grid-connected
Battery Systems with Reactive Power Control Capability,"  in Proc. of  North American Power Symposium (NAPS), September 2017, Morgantown, WV.


\item [C2] Z. Taylor, {\bf H. Akhavan-Hejazi} ,  E. Cortez, L. Alvarez, S.  Ula, M.  Barth, H. Mohsenian-Rad, ``Battery-assisted distribution feeder peak load reduction: Stochastic optimization and utility-scale implementation." In Proc. of Power and Energy Society General Meeting (PESGM), July 2016, Boston, MA.



\item [C3] {\bf H. Akhavan-Hejazi}, B. Asghari, R. Sharma, ``A joint bidding and operation strategy for battery storage in multi-temporal energy markets," in Proc. of the \emph{IEEE PES Innovative Smart Grid Technologies Conference}, Washington, DC, Feb. 2015. 

\item [C4] H. Darvishi, A. Darvishi, {\bf H. Akhavan-Hejazi}, ``Integration of demand side management in security constrained energy and reserve market," in Proc. of the \emph{IEEE PES Innovative Smart Grid Technologies Conference}, Washington, DC, Feb. 2015.

\item [C5] {\bf	H. Akhavan-Hejazi}, H. Mohsenian-Rad, A. Nejat, ``Developing a test data set for electric vehicle applications in smart grid research," in Proc. of the \emph{IEEE Vehicular Tec. Conf.}, Vancouver, BC, 2014. 

\item  [C6]	Chenye Wu, { \bf H. Akhavan-Hejazi}, H. Mohsenian-Rad, Jianwei Huang ``PEV-based P-Q Control in Line Distribution Networks with High Requirement for Reactive Power Compensation", in Proc. of the \emph{IEEE PES Innovative Smart Grid Technologies Conference}, Washington, DC, Feb. 2014. 

\item [C7]	{\bf H. Akhavan-Hejazi}, H. Mohsenian-Rad, ``A Stochastic Programming Framework for Optimal Storage Bidding in Energy and Reserve Markets," in Proc. of the \emph{IEEE PES Innovative Smart Grid Technologies Conference}, Washington, DC, Feb. 2013.

\item [C8] {\bf	H. Akhavan-Hejazi}, Z. Bahar, H. Mohsenian-Rad, ``Challenges \& Opportunities in Large-Scale Deployment of Automated Energy Consumption Scheduling in Smart Grid," in Proc. of \emph{the IEEE Conf. on Smart Grid Communications}, Tainan, Taiwan, Oct. 2012. 

\item [C9] {\bf	H. Akhavan Hejazi}, M. Abedi, H. R. Mohabati, M. Hajizade, ``A New Approach for Fast Identification of Distribution System Dynamic Model from Measured Data," in Proc. of the \emph{International Conference on Clean Electrical Power}, Ischia, Italy, 2011. 

\item [C10] {\bf H. Akhavan-Hejazi}, M. Hejazi, G. Gharehpatian, M. Abedi, ``Distributed Generation Site and Size Allocation Through a Techno Economical Multi-objective Differential Evolution Algorithm," in Proc. of \emph{the IEEE Power \& Energy Int. Conference}, Kuala Lumpur, Malaysia, 2010.




\end{rSection}



%%%%%%%%%%%%%%%%%%%%%%%%%%%%%%%%%%%%%%%%%%%%%%%%%%%%%%%%%%%%%%%%%%%%%%%%%%%%%%%%%%%%%%%%%%%%%%%%%%%%%%%%%%%%%
%\begin{rSection}{Selected Proposal Writing Experience}
%\textbf{Summary:}  I was actively involved in developing  the technical documents for several proposals, including one or more of  the Technical Volume, Project Narrative, Scope of Work, Project Time-line, Fact-sheet, etc. The list below is a non-exhaustive proposals during my PhD. and Postdoc.
%\begin{itemize}
%
%\item  \emph{UCOP-2018 UC-NLCRT:} UC-Lab Center for Electricity Distribution Cybersecurity. (\textbf{Awarded: \$3M})
%
%\item  \emph{CEC-GFO-16-304:} Internet of Things and Ubiquitous Sensing in University Building Energy Management; Design Optimization and Technology Demonstration. (\textbf{Awarded: \$2.5M})
%
%\item  \emph{CEC-GFO-16-309:} Advanced Integrated Building Energy Management Technology Demonstration in a Permanent Supporting Housing Facility. (\textbf{Awarded: \$2.1M})
%
%\item  \emph{CEC-EISG-EISG-13-04:} PEV-Based Active and Reactive Power Compensation in Distribution Networks: Design Optimization and Technology Demonstration.  (\textbf{Awarded: \$100K})
%
%\item  \emph{CEC-GFO-16-303-G3:} Optimized Large Vehicle Battery Recycling for Grid Integrated Applications. (Declined: \$1.0M)
%
%\item \emph{CEC-GFO-15-313-G3:} Exploiting PMU Data to Enable Bi-directionality, Enhance Reliability, and Improve Efficiency in California Distribution Feeders. (Declined: \$1.7M)
%
%\item \emph{RPU-EI-14:} Monitoring and Control of PVs, Battery Storage Systems, and EV Chargers at a 12 kV Industrial Substation Feeder Level. (\textbf{Awarded: \$100K})
%
%\item \emph{RPU-EI-15:} Exploiting PMU Data at RPU's 12 kV Industrial Feeder; Innovative Data Analytics and Optimal Energy Resource Operation. (Declined: \$100K)
%
%\item \emph{DoE-FOA-1616:} Learning-Enhanced Algorithms, DER Synthesis. (Declined: \$1.5M)
%
%\item \emph{DoE-FOA-1493:} Tackling Market Economics and Grid Reliability Risk Tradeoffs in Market Aggregation of Demand Resources; A Decentralized Approach Driven by Big Data. (Declined: \$300K)
%
%\end{itemize}
%
%
%\end{rSection}
%

%%%%%%%%%%%%%%%%%%%%%%%%%%%%%%%%%%%%%%%%%%%%%%%%%%%%%%%%%%%%%%%%%%%%%%%%%%%%%%%%%%%%%%%%%%
%%------------------------------------------------------------------------------------
%\begin{rSection}{ Teaching Interests}
%
%\begin{tabular}{ @{} l  l   } 
%\vspace{0.16cm}
%
%System analysis: & \begin{tabular}{l}  Engineering circuit analysis, power systems analysis,\\  energy systems operation \& planning.   \end{tabular} \\
%\vspace{0.16cm}
%
%Optimization and Control:   & \begin{tabular}{l} Linear systems, control theory, optimization theory, stochastic optimization.   \end{tabular} \\
%\vspace{0.16cm}
%
%Data analysis:&  \begin{tabular}{l} Statistical analysis,  state estimation. \end{tabular} \\
%
%\end{tabular}
%
%
% \vspace{0.16cm}
%
%\end{rSection}





%
%%%%%%%%%%%%%%%%%%%%%%%%%%%%%%%%%%%%%%%%%%%%%%%%%%%%%%%%%%%%%%%%%%%%%%%%%%%%%%%%%%%%%%%%%%%%%%
%\begin{rSection}{Teaching Experience}
%
%\begin{rSubsection}{Teaching Assistant }{2014 - 2016}{Department of Electrical \& Computer Engineering , University of California at Riverside}{}
%\item [] EE231: Convex Optimization in Engineering Applications (Graduate Course), Winter 2014
%\item [] EE232: Introduction To Smart Grid (Graduate Course), Winter 2016
%%\begin{innerlist}
%\end{rSubsection}
%
%
%\begin{rSubsection}{Teaching Assistant }{2010 }{Department of Biomedical Engineering, Amirkabir University of Technology }{}
%\item [] DC Electrical Machines (Under-graduate Course), Fall 2010
%%\begin{innerlist}
%\end{rSubsection}
%
%\end{rSection}
%


%%%%%%%%%%%%%%%%%%%%%%%%%%%%%%%%%%%%%%%%%%%%%%%%%%%%%%%%%%%%%%%%%%%%%%
\begin{rSection}{Professional Services}
\item  Committee Chair,  UCR Conference on  Energy Storage Technologies and Applications, 2018-2019.
\item  Co-Chair of Control and Operation Symposium, IEEE Smart Grid Communications Conference, 2018.
\item  Technical Program Committee  Member, IEEE Smart Grid Communications Conference, 2014-2017.
\item   Technical Program Committee Member, IEEE Global Communications Conference, 2016.
\item Technical Program Committee Member, Smart Grid Inspired Future Technologies Conference, 2016.
\item  Technical Program Committee Member, IEEE Vehicular Technology Conference, 2014.
\item  Reviewer for IEEE Transactions on Power Systems,  Smart Grid, and Sustainable Energy.

\end{rSection}


\begin{rSection}{Fellowships \& Awards}

\item [] Dissertation Year Award, University of California, Riverside, Graduate Division, 2015

\item [] Dean's Distinguished Fellowship Award, University of California, Riverside, Graduate Division, 2012

\item [] Dean’s Sybil Harrington Living Trust Fellowship, Texas Tech University, Graduate School, 2011

\end{rSection}


%%%%%%%%%%%%%%%%%%%%%%%%%%%%%%%%%%%%%%%%%%%%%%%%%%%%%%%%%%%%%%%%%%%%%%%%%%%%%%%%%%%%%%%%%%%%%%%%%%%%%%%%%%%%
\begin{rSection}{ Proposal Writing Experience}
\textbf{Summary:} During my PhD. and Postdoc, I was actively involved in developing  the technical documents, such as Technical Narrative, Scope of Work, etc. for several proposals. Here is a select list:
\begin{itemize}

\item  \emph{UCOP-2018 UC-NLCRT:} UC-Lab Center for Electricity Distribution Cybersecurity. (\textbf{Awarded: \$3M})

\item  \emph{CEC-GFO-16-304:} Internet of Things and Ubiquitous Sensing in University Building Energy Management; Design Optimization and Technology Demonstration. (\textbf{Awarded: \$2.5M})

\item  \emph{CEC-GFO-16-309:} Advanced Integrated Building Energy Management Technology Demonstration in a Permanent Supporting Housing Facility. (\textbf{Awarded: \$2.1M})

\item  \emph{CEC-EISG-EISG-13-04:} PEV-Based Active and Reactive Power Compensation in Distribution Networks: Design Optimization and Technology Demonstration.  (\textbf{Awarded: \$100K})

\item  \emph{CEC-GFO-16-303-G3:} Optimized Large Vehicle Battery Recycling for Grid Integrated Applications. (Declined: \$1.0M)

\item \emph{CEC-GFO-15-313-G3:} Exploiting PMU Data to Enable Bi-directionality, Enhance Reliability, and Improve Efficiency in California Distribution Feeders. (Declined: \$1.7M)

\item \emph{RPU-EI-14:} Monitoring and Control of PVs, Battery Storage Systems, and EV Chargers at a 12 kV Industrial Substation Feeder Level. (\textbf{Awarded: \$100K})

\item \emph{RPU-EI-15:} Exploiting PMU Data at RPU's 12 kV Industrial Feeder; Innovative Data Analytics and Optimal Energy Resource Operation. (Declined: \$100K)

\item \emph{DoE-FOA-1616:} Learning-Enhanced Algorithms, DER Synthesis. (Declined: \$1.5M)

\item \emph{DoE-FOA-1493:} Tackling Market Economics and Grid Reliability Risk Tradeoffs in Market Aggregation of Demand Resources; A Decentralized Approach Driven by Big Data. (Declined: \$300K)

\end{itemize}


\end{rSection}


\end{document}
